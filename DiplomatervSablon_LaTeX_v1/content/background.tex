% !TeX spellcheck = en_GB
% !TeX encoding = UTF-8
% !TeX program = xelatex
%----------------------------------------------------------------------------
\chapter{Background}
%----------------------------------------------------------------------------

\section{Graph Neural Networks}

	

	\subsection{Neural Networks}
	
	Neural networks are an especially interesting and useful area of machine learning and data analysis: as referenced in their name, they were created in resemblance of the human brain's structure, mimicking the interconnected neurons. In recent years they became the flagship of AI research because of their ability to work on incredibly large datasets effectively and produce never seen before results. 
	
	The basis of a neural network are neurons which sum up incoming "signals" multiplied by weights and apply an activation function to their output. The network learns by updating these weights and biases based on the training data, to match the desired output to each piece of input. The activation function adds non-linearity to the model, making it capable of learning more complicated relationships in the data.
	
	%TODO cite
	From the most fundamental concepts of the original perceptron and the breakthrough idea of using backpropagation to minimize a loss function, neural networks became a widespread and varied phenomenon: there are architectures for different types of data optimized for different kinds of tasks; these models have proved to be applicable in almost any area.
	
	
	%TODO extrába itt valami a backpropról?
	
	Neurons are typically organized into layers in a neural network and the models usually benefit from having a large number of layers: modern models for more complex tasks can get very 'deep' which lead to the popularization of the term deep learning when talking about such models. But since having more layers makes a model more computationally intensive, leading to higher costs and slower inference, researchers are often looking to prune models or find architectures that deliver similar results while having a lower parameter count.
	
	A very active and widely used branch of neural networks are convolutional neural networks (CNNs). These are most commonly used on image data and work by convolving learnable kernels with the input to extract features. This is useful for capturing spacial relationships allowing for better pattern detection while also greatly decreasing the number of trainable parameters compared to a fully connected feed-forward network.
	
	%TODO valami transition már itt a gráfokhoz?
	
	\subsection{Working on Graph Data}
	
	Graphs are a very versatile mathematical concept because of the way they lend themselves very neatly as a generalization. They have great expressive power when it comes to describing relations, groups and things with a rich inner structure. We can find graph-based datasets in a variety of domains: molecules in chemistry, interaction graphs in social sciences, knowledge graphs and computer networks. The level of abstraction they provide makes them suitable for use in many fields, as connections between items or molecules or people are vital in understanding complex natural systems.
	
	Graphs are mathematically defined as a set of nodes (or vertices) and edges (or links) which run between two nodes. They can be categorized by their edge structure: directed graphs specifically code the information of a starting and end node, while undirected graphs do not. In terms of graph neural networks this is a very important distinction as it fundamentally alters how the network's meaning. Certain types of networks are designed with a specific type of graph in mind and might need special normalisation.
	
	Certain types of graphs with special properties can be especially interesting in certain areas. Trees (which contain no cycles) can be useful for describing containment hierarchies or sentence structures, for example. Based on semantic meaning we can also talk about heterogeneous graphs were nodes may represent entirely different things, purchase or interaction graphs in a recommender system where a vertex could be a user or an item to be purchased. In this case it is important to make this information available to the network.
	
	A graph dataset often codes much more information then just the mathematical structure itself. Ideally some information is available of each node: 
	
	One area that is particularly rich with such examples is biology and medicine. Interactions between drugs, relationships between species and contact tracing in epidemiology are all important facets that can benefit from the ability to analyse 
	graph datasets. We can also find interesting networks to analyse inside of organisms: the focus of this thesis is the analysis of network formed by brain regions and the interactions that happen in the human brain.
	
	Since graphs can be used to represent complex structures and hierarchies, where this inner construction is more pronounced and relevant then in other cases, they require specialized methods when it comes to machine learning and neural networks.
	
	
		\subsubsection{Common tasks}
		
		The motivation behind working with graph datasets is of course to perform some sort of inference using the patterns extracted from the training data. Since these datasets are diverse both in semantics and in structure this could mean a lot of different types of tasks. In this section the most common of these will be described with examples and commonly used methods for solving them.
		
		\textbf{Node prediction.}
		
		\textbf{Graph prediction.}
		
		\textbf{Edge prediction.}
		
		\textbf{Clustering.}
	
	\subsection{Graph Neural Network Types}
	
	
		\subsubsection{Graph Convolutional Networks}
		
		Graph Convolutional Networks (GCNs) expand on the idea of CNNs, which are commonly used in image processing. Observing from a graph perspective, an image can be considered a very special case of a graph, where pixels are nodes and their neighbours can be determined from the grid. This type of network generalizes the concept of local convolutions from the image domain to general graphs.
		
		\subsubsection{Graph Attention Networks}
	
\section{Medical data}

	We can call anything medical data that relates to human healthcare in one way or another. This most commonly means data collected by healthcare institutions about patients using sensor equipment, but it could also be information about a patient's habits or general environment, data from drug test trials or location data in case of epidemiological contact tracing. 
	
	This kind of data is often very personal and regarded as sensitive data: people are entitled to equal treatment regardless of medical status and as such information about the health status of an individual is protected. Many patients do not want employers or other third parties who could use such information against them to know details about their conditions and their treatment.
	
	%TODO add law stuff maybe?
	
	For this reason doctors are required to not give out patient information unless it is strictly necessary and/or they have the informed consent of the person. This puts medical researchers in an interesting position; medical researchers using machine learning even more so. Data collected specifically for experiments where subjects can consent to their data being used is a straightforward situation, but there is data collected every day, worldwide in hospitals that could be very useful for furthering medicine, but doing so poses data privacy concerns. This is especially important in case of machine learning and neural network research as these disciplines require a very large amount of data that is very hard to collect using only organized experiments.
	
	 %TODO some ethics stuff 
	 
	 %TODO dataset anonimisation

	\subsection{Medical imaging}
	
	Non-invasive medical imaging techniques, such as MRI, X-ray, and CT scans, are essential tools in medicine, allowing physicians to understand the internal structures and functions of the body without the need for surgery. These techniques are invaluable for their ability to produce high-resolution images of different types of tissues, making it ideal for assessing the state and function of internal organs. 
	
	
	
	%TODO valami historyja a dolognak meg egy kicsi a többi módszerről	
	
	\subsection{MRI and fMRI imaging}
	
	Magnetic Resonance Imaging (MRI) produces its high-fidelity images, while using no ionizing radiation, making possible its repeated use in patients. 
	
		\subsubsection{Technical background}
		
		
		
		\subsubsection{Processing}
		
		
		%TODO ide jobb szó a clinical helyett?
		\subsubsection{Clinical significance }