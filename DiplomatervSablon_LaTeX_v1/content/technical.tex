% !TeX spellcheck = en_GB
% !TeX encoding = UTF-8
% !TeX program = xelatex
%----------------------------------------------------------------------------
\chapter{Technical background}
%----------------------------------------------------------------------------

\section{Deep Learning Frameworks}

	In my work I used Python both for implementing machine learning solutions and for creating pipelines; downloading, sorting and preparing data, as well as various scripting needs. For the purpose of organizing the used Python packages and ensuring a consistent environment I have used a virtual conda environment created with miniconda.
	
	For machine learning tasks I have employed various libraries commonly used for such tasks to avoid reimplementing standard solutions, both in support of my own final solutions and for creating baseline solutions for comparison and for data exploration and visualization purposes.

	\subsection{Pytorch}
	
	\subsection{DGL}
	
	\subsection{Sklearn}
	
	\subsection{Tsfresh}
	
	\subsection{Xgboost}
	
	\subsection{Supporting packages}

\section{Using fMRI data}

	\subsection{Data formats}
	
	\subsection{Pre-processing}
	
	\subsection{Extracting ROI Values and Connectomes}