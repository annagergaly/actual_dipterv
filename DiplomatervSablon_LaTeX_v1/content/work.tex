% !TeX spellcheck = en_GB
% !TeX encoding = UTF-8
% !TeX program = xelatex
%----------------------------------------------------------------------------
\chapter{Methods and Implementation}
%----------------------------------------------------------------------------

%TODO jobb nevet kitalálni ennek?
\section{Graph Neural Networks on Connectomes}

	\subsection{Dataset}
	
	%TODO find citation og ABIDE
	The ABIDE (Autism Brain Imaging Data Exchange) dataset focuses on the furthering research into ASD (autism spectrum disorder) through neuro-imaging and neuroscience. The dataset contains in total 1112 resting state fMRIs of subjects both with ASD (539 individuals) and typically developing controls (573 individuals). 
	
	The imaging data has been collected by 16 institutions, who collaborated to create a publicly accessible anonymised dataset to allow the broader scientific community to take part in ASD related research while preserving the privacy of the participants. As such, datasets do not contain any protected health information. 
	
	%TODO cite preprocessed 
	Since fMRI preprocessing techniques are very complex image processing operations and heavily resource intensive, an initiative also formed to provide a preprocessed version of the dataset. 
	
	Since there is not a widely accepted best practice method for preprocessing fMRI images, the participating five teams have all used their preferred methods to clean the data. The participants have also provided a ROI activation timeseries calculated using 7 different brain atlases from each type of the preprocess method. 
	
	The very close ratio of affected vs control subjects (539 to 573) means the dataset is well-balanced in terms of a classification task on the diagnosis of subjects. 
	
	\subsection{Methodology}
	
	%TODO cite
	In the paper \cite{wang2021graph} by Wang et al. researchers worked with the ABIDE dataset to create a graph convolutional network based architecture for predicting the diagnosis of a patient based on the resting state fMRI. 
	
	\subsection{Results}
	
	\subsection{Conclusions}

\section{Connectome-based diffusion}

	\subsection{Dataset}
	
	\subsection{Methodology}
	
	\subsection{Results}
	
	\subsection{Conclusions}

\section{Minimising device-to-device differences}

	\subsection{Dataset}
	
	\subsection{Methodology}
	
	\subsection{Results}
	
	\subsection{Conclusions}